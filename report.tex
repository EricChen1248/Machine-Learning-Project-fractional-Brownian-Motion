\documentclass[a4paper, 12pt]{article}
\usepackage{cite}
\usepackage{multicol, caption}
\usepackage{titling}
\usepackage{graphicx}
\usepackage{tabularx}
\usepackage{subcaption}
\usepackage{url}
\usepackage{amsmath}

\usepackage{listings}
\usepackage{color}

\definecolor{dkgreen}{rgb}{0,0.6,0}
\definecolor{gray}{rgb}{0.5,0.5,0.5}
\definecolor{mauve}{rgb}{0.58,0,0.82}

\lstset{frame=tb,
  language=python,
  aboveskip=3mm,
  belowskip=3mm,
  showstringspaces=false,
  columns=flexible,
  basicstyle={\small\ttfamily},
  numbers=none,
  numberstyle=\tiny\color{gray},
  keywordstyle=\color{blue},
  commentstyle=\color{dkgreen},
  stringstyle=\color{mauve},
  breaklines=true,
  breakatwhitespace=true,
  tabsize=3
}

\newenvironment{Figure}
  {\par\medskip\noindent\minipage{\linewidth}}
  {\endminipage\par\medskip}

%加這個就可以設定字體
\usepackage{fontspec}
%使用xeCJK,其他的還有CJK或是xCJK
\usepackage{xeCJK}
\usepackage[margin=0.6in]{geometry}

%字型的設定可以使用系統內的字型,而不用像以前一樣另外安裝
\setCJKmainfont{DFKai-SB}
\author{}    
\newcommand*{\citenumfont}{}

% Reducing Distance of title and date
\setlength{\droptitle}{-2cm}
\title{\textbf{Fractional Brownian Motion}}
\renewcommand\maketitlehookc{\vspace{-3.5em}}


\begin{document}
    \maketitle     
    \begin{center}
        \begin{tabular}{ccc}
            陳君彥, b04703091 & 劉育嘉, b06902008 & 黃柏豪, b06902124 \\
            b04703091@ntu.edu.tw &
            b06902008@ntu.edu.tw &
            b06902124@ntu.edu.tw
        \end{tabular}
    \end{center}

    \section*{Division of Work}
        陳君彥, b04703091: 
        \begin{itemize}
            \item Data initial processing and calculation tool creation.
            \item Initial creation and testing of XG-boost, LightGBM, and Random Forest Models
        \end{itemize}
        劉育嘉, b06902008:
        \begin{itemize}
            \item Indepth testing of tree based methods
            \item Testing and Creation of select feature models.
        \end{itemize}
        黃柏豪, b06902124:
        \begin{itemize}
            \item Creation and testing of neural network based models.
            \item Creation and testing of blending based models.
        \end{itemize}

        \begin{figure}
            \includegraphics[width=\linewidth]{images/score.png}
            \captionof{figure}{Submission scores of our models.}
            \label{score}
        \end{figure}

        \section{Introduction}
        The goal of this project is to reverse learn model parameters used to simulate a fractional Brownian Motion \cite{fBM} simulation. The parameters used to run this simulation were alpha, mesh size, and penetration rate. We used 2 methods of evaluation, one with "Weighted Mean Absolute Error" (WMAE), and one with "Normalized Absolute Error" (NAE)
        \[WMAE(Y, \hat{Y}) = \frac{1}{n_{samples}}\sum^{n_{samples}}_{i=1}\sum^{3}_{j=1}w_j\mid y_{ij}-\hat{y}_{ij} \mid, NAE(Y, \hat{Y}) = \frac{1}{n_{samples}}\sum^{n_{samples}}_{i=1}\sum^{3}_{j=1} \frac{\mid y_{ij}-\hat{y}_{ij} \mid}{y_{ij}}\]
        
        \newpage

        \begin{multicols}{2}
        \section{Features}
        \subsection{Original Features}
        Our training dataset consists of 47,500 simulations with 10,000 features each. The first 5,000 features represented the mean-square displacements (MSD) of our particles ordered from time = [1,5000]. The second 5,000 features were 50 sets of velocity auto-correlations (VAC) calculated in different methods, ordered by the early VAC being more representative of instataneous velocity, with later VAC being closer to average velocity. Each sets consists of 100 calculation using the respective VAC, with time intervals from t = [1,100].
        
        Our testing sets consists of 2,500 simulations, half of which is tested in the public score, and half hidden in the private score.

            \subsubsection{Feature Importance}
                In the interest of reducing the time need to train our models, we took a look at the feature importance determined by our earlier models and selected those that had an importance higher than a given threshold. Through several rounds of testing, we found $10^{-4}$ to perform the best. This allowed us to greatly increase the training iteration count and deepen our original model, while giving us increased accuracy with much lower training times.
                
                Something of particular to note is the high reliance of a select few features for each parameter. Due to the lack of professional knowledge in this field and the lack of raw data, we could do little to take advantage of this except to reduce the dimensionality. However we expect that further studies can be improve by generating more features that are centered around these specific features.

            \subsection{Additional Features}
                To gather more physic based features, we also extrapolated some of our own data.

                \begin{Figure}
                    \centering
                    \includegraphics[width=\linewidth]{images/importance_1.png}
                    \captionof{figure}{Feature importance of penetration rate.}
                    \label{alphafeat}
                \end{Figure}
                \begin{Figure}
                    \centering
                    \includegraphics[width=\linewidth]{images/importance_2.png}
                    \captionof{figure}{Feature importance of mesh size.}
                    \label{meshfeat}
                \end{Figure}
                \begin{Figure}
                    \centering
                    \includegraphics[width=\linewidth]{images/importance_3.png}
                    \captionof{figure}{Feature importance of alpha.}
                    \label{penfeat}
                \end{Figure}

                \subsubsection{Average and MinMax}
                    We were interested in having our model be less overfitting and took feature based methods to reduce the feature count. We tried taking the average of every 10 features, which would result in the mean displacement or velocity over a larger time period. We also tried taking the minimum and maximum of each 10 features, which in essence was is a form of data pooling.

                \subsubsection{Square Error on Linear Regression}
                    We assume a particle without external force will go straight forward with constant speed, so the force generated by collision might have something to do with the difference between a linear line and its real value. Therefore, we performed a linear regression compute the variance of those 5000 time intervals, substituting them for the original 5000 features.

                \subsubsection{Alpha feedback}
                    Our models performed exceptionally well on alpha, and so we decided to feedback the results of our third parameter prediction back in as a feature for our other two predictions, giving a slight boost to performance.

        \section{Individual Model Experiments}        
            \subsection{Tree based models}
                \subsubsection{Random Forest}
                We used the models provided in the python package Sci-kit Learn \cite{scikit-learn} for our random forest, with bootstrap enable, estimator count of 50, and a max depth of 9. The following shows the results of our experiments.
                \begin{center}
                    \begin{tabular}{c|ccc}
                        Tree count & Penetration & Mesh Size & Alpha\\
                        \hline
                        50 & 65.72 & 31.92 & 1.96 \\
                        100 & 64.74 & 30.69 & 1.81 \\
                        200 & 64.06 & 29.93 & 1.74 \\
                        300 & 61.02 & 27.20 & 1.54 \\
                        500 & 58.67 & 25.10 & 1.40 \\
                        1250 & 55.07 & 22.40 & 1.23
                    \end{tabular}
                    \captionof{table}{Performance on track 1 for random forest (Validation)}
                    \label{randForTab}
                \end{center}

                As seen in Table \ref{randForTab}, we achieve moderate performance by simply throwing in the original features into a random forest model. The performance gradually increased with each additional tree count, though the time required to train them starting to get out of hand. Each tree required around 1.5 minute to train when parallelized, meaning training each model for all three features required over 3 hours for each incrementation. Hence we tried it out with a reduced feature set selected using the previously trained model's feature importances.

                
                \begin{center}
                    \begin{tabular}{c|ccc}
                        Tree count & Alpha & Mesh Size & Penetration\\
                        \hline
                        50 & 67.31 & 30.69 & 3.13 \\
                        100 & 67.42 & 30.60 & 3.12 \\
                        200 & 67.35 & 30.65 & 3.11 \\
                        300 & 67.35 & 30.64 & 3.11 \\
                    \end{tabular}
                    \captionof{table}{Performance on track 1 for random forest with reduced input. (Validation)}
                    \label{randForRedTab}
                \end{center}

                However, as seen in Table \ref{randForRedTab}, the reduced feature set performed poorly, while giving us a very low training time, it failed to beat even the 50 tree count on the regular RF model at 300 trees. We gave up and tried a different approach, reducing the dataset by picking averages of every 10 features.

                
                \begin{center}
                    \begin{tabular}{c|ccc}
                        Tree count & Alpha & Mesh Size & Penetration\\
                        \hline
                        50 & 64.57 & 23.98 & 2.37 \\
                        100 & 64.56 & 23.94 & 2.36 \\
                        200 & 64.56 & 23.86 & 2.36 \\
                        300 & 64.60 & 23.85 & 2.35 \\
                    \end{tabular}
                    \captionof{table}{Performance on track 1 for random forest with averaged features. (Validation)}
                    \label{randForAvgTab}
                \end{center}

                The results as shown in Table \ref{randForAvgTab} shows that this performed better than the feature selection model, and slighly better than the original model, with a combined score of 90.81 on track 1 and 2.29 on track 2. However, we notice that stop in decrese of the validation error, and recognized that further training would do the model no good.

                \subsubsection{XGBoost}
                    We tried the very popular XGBoost \cite{xgboost} often used in machine learning competitions. XGBoost is a gradient boosting tree model that has an emphasize on reduced tree depth and higher parallelism. This allows us to train a model much faster and allows easier batching on the massive dataset. 

                    
                \begin{center}
                    \begin{tabular}{l|cc}
                        Notes & Track 1 & Track 2\\
                        \hline
                        base & 137.23 & 6.22 \\
                    \begin{tabular}{@{}l@{}}+ filtered data \\ \& alpha feedback \end{tabular} & 48.86 & 0.66 \\
                        + auto tree method & 43.49 & 0.60 \\
                        + tweedie regularizer & 41.70 & 0.38 \\
                        + boost round = 5 & 40.85 & 0.33 \\
                    \end{tabular}
                    \captionof{table}{Performance of xgboost with each additional parameter tuning (Validation).}
                    \label{xgboost}
                \end{center}
                    As seen in table \ref{xgboost}, xgboost relies heavily on custom tuned parameter, with the untuned model performing much worse than others. After some tuning and using filtered data, our model provided a very respectable result on the validation set. After submitting the prediction, we found out that the final rounds of tuning had led to the model being slightly overfitted, with the submission score being worse than the previous round. We thus stopped our experiments on xgboost here and moved on to a different model.
                \subsubsection{LightGBM}
                    LightGBM \cite{lightgbm} is a very new contender to the scene, with a focus on limiting leaf count rather than xgboost's method of limiting tree depth.

                    In our instant, lightgbm performs much better with base parameters, with learning rate set to 0.01. As seen in Table \ref{lightgbmVal}, it is already in clear competition with xgboost. We also went through the same process of using the predicted alpha to feed back into the model, as well changing to use 'dart' as a boosting method with more iterations. It can be seen from Table \ref{lightgbmSub} that the submission score is also dropping at a steady rate, but then shows signs of overfitting with the last submission, we then started performing regularization to combat the issue.

                    \begin{center}
                        \begin{tabular}{l|cc}
                            Notes & Track 1 & Track 2\\
                            \hline
                            base & 56.80 & 1.52 \\
                            + alpha feedback & 55.87 & 1.20 \\
                            + 1000 iterations + dart & 38.25  & 1.07 \\
                            + 2500 iterations & 30.84 & 0.36 \\
                        \end{tabular}
                        \captionof{table}{Validation performance of lightgbm with each additional parameter tuning.}
                        \label{lightgbmVal}
                    \end{center}

                    We tried using the square error on linear regression as a feature, and averaging them out, cutting our feature set down to one tenth of the size, and as seen in Table \ref{lightgbmsquared}, shows that the overfitting issue has been somewhat allieviated.

                    \begin{center}
                        \begin{tabular}{l|cc}
                            Notes & Track 1 & Track 2\\
                            \hline
                            base & 60.76 & 2.94 \\
                            + alpha feedback & 59.87 & 2.10 \\
                            + 1000 iterations + dart & 48.23 & 0.92 \\
                            + 2500 iterations & 50.48 & 0.86 \\
                        \end{tabular}
                        \captionof{table}{Submission performance of lightgbm with each additional parameter tuning.}
                        \label{lightgbmSub}
                    \end{center}

                    \begin{center}
                        \begin{tabular}{l|cc}
                            Notes & Validation & Submission \\
                            \hline
                            500 iterations & 45.52 & 49.17 \\
                            1000 iterations & 43.27 & 48.13 \\
                        \end{tabular}
                        \captionof{table}{Submission performance of lightgbm with regularization.}
                        \label{lightgbmsquared}
                    \end{center}

            \subsection{Neural network models}
                We used the neural network model found in keras \cite{keras} to predict our three parameters. As the number of features is too big for a neural network, the features were first passed through an autoencoder of size 5000-2048-1024 and train with 'linear' activation on a 5000-2048-512-64-32-1 model.
            \begin{lstlisting}
def encoder(x, 
    activation_list = ['relu', 'relu'], 
    layer_list = (2048, 1024), # dimension for each layer
    name = 'temp', # to specify model file
    use_old = False): # load existing model or not
            \end{lstlisting}

            However, our model (loss='mse', optimizer = 'adam', batch size = 64, epochs = 10), as shown in Table \ref{nnresult}, performed poorly alone.
                
            \begin{center}
                \begin{tabular}{l|cc}
                    Notes & Track 1 & Track 2 \\
                    \hline
                    In-Sample Error & 120.19 & 3.45 \\
                    Out-of-Sample Error & 121.47 & 3.52 \\
                \end{tabular}
                \captionof{table}{Performance of the neural network model.}
                \label{nnresult}
            \end{center}
            
            We attempted to improve it by reducing the layer count, while changing the activation method to use 'relu' and arrived at the results shown in table \ref{nnrelu}. While not the best, we deemed it good enough to be used in our final method.

            \begin{center}
                \begin{tabular}{l|cc}
                    Notes & Track 1 & Track 2 \\
                    \hline
                    In-Sample Error & 98.15 & 1.82 \\
                    Out-of-Sample Error & 100.15 & 1.85 \\
                \end{tabular}
                \captionof{table}{Performance of the improved neural network model.}
                \label{nnrelu}
            \end{center}

            \subsection{Blending}
                Our last methods involved the blending of our previously successful models. The first one we tried as the mixing of xgboost into our neural network. We already showed that xgboost provides a very good performance that can be extended upon. We used the autoencoded features as our inputs with weights initialized to 0, and xgboost predictions as a second input with weights initialized to 1, and the same layers as used before. As shown in table \ref{xgbnn}, this is our best results so far, though we suspect it may be somewhat a lucky guess due to the slight overfitting that occurs.

                \begin{center}
                    \begin{tabular}{l|cc}
                        Notes & Track 1 & Track 2 \\
                        \hline
                        In-Sample Error & 35.56 & 0.33 \\
                        Out-of-Sample Error & 44.79 & 0.45 \\
                        Submission & 47.71 & 0.71
                    \end{tabular}
                    \captionof{table}{Performance of xgboost + NN}
                    \label{xgbnn}
                \end{center}

                We tried to replicate it further, using a uniform blending of our top three models, xgboost, NN + xgboost, and lightgbm. As shown in table \ref {uniform}, the blending has made our model much more stable, providing better performance while being less overfitted than our previous model.
                \begin{center}
                    \begin{tabular}{l|cc}
                        Notes & Track 1 & Track 2 \\
                        \hline
                        Validation & 39.63 & 0.42 \\
                        Submission & 46.87 & 0.65
                    \end{tabular}
                    \captionof{table}{Performance of Uniform Blending Model}
                    \label{uniform}
                \end{center}
            \end{multicols}
        \section{Conclusion}
                In this paper, we tested a multitude of methods to predict the parameters used to generate a fractional Brownian Motion simulation. We used methods that reduced the feature set size of each entry, including feature importance as determined by preliminary models, as well as mathematical equations to reduce the dimensionality. This reduction allowed us to train the models much further than was previously feasible. We also generated our own features, using physical features like the MSE on linear regression or a feedback of our own predicted alpha. These features were ran through multiple different models that provided different insights.
                
                First, our pure Random Forest Models performed moderately, and showed a clear sign of the error trending downwards with each iteration. It's downfall is the massively long time it took to train each iteration, making it unfeasible for such a large dataset. 
                
                We took a look at improved tree models, such as the gradient boosted xgboost and lightgbm. These models showed the benefit of tree based models in being quite performant, while also being much faster to train than the random forest models. This allowed us to test many more parameters and feature inputs than before, giving us a better heading to work off of.

                Next, we also tried neural network based models. Alone, our NN model performed moderately. The problem with neural network being that it is rather hard to improve iteratively. One set of parameters or settings may work better than another, but a combination of them may perform even worse than either. So we tried giving our neural network a jumpstart by also giving it the predicted results of our xgboost models as a starting point. This gave us a very good accuracy, and showed that there was more information that could be extracted from the xgboost. This model however, was also suspicious of being incredibly overfitting, with our high accuracy possibly due to luck.

                Our final model is the uniform blending of the three models, NN + xgboost, xgboost, and lightgbm. This blending stabilized our results, leading to a lower validation accuracy, but a higher test accuracy.  We arrive at the conclusion that the blending of multiple successful models is one of the best way to achieve a good and stable way of creating a machine learning model.

\vskip 5cm
\bibliography{report}{}
\bibliographystyle{ieeetran}

\end{document}
